\documentclass{memoir}
\begin{document}
\title{Wednesday Exercises}
\date{January 23 2019}
\maketitle

\section*{\Huge{Exercise 1}}
\begin{Large}
Create a function called \textbf{simpleInterest} that calculates a total after a
given amount of time at a given rate. \\
The \textbf{parameters} for this function are as follows: \\
\textbf{start}: the amount of money you are starting with \\
\textbf{years}: the number of years that iterest is to be applied to the
starting amount
\\
\textbf{rate}: the rate at which interest is gained \\
The function should return a float, rounded off to 2 decimal places
\end{Large}
\\

\section*{\Huge{Exercise 2}}
\begin{Large}
Create a function called \textbf{minCoins} that finds the minimum number of coins required to equal a given sum\\
The \textbf{parameters} for this function are as follows:
\\
\textbf{amount}: the sum you are trying to create. This should be a float. Ex: \$5 = 5.00
\\
The function should return an integer representing the minimum nuber of coins
\end{Large}
\end{document}